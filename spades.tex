\documentclass{article}
\usepackage{graphicx} % Required for inserting images
\usepackage{grbbridge}
\usepackage[utf8x]{inputenc} % Включаем поддержку UTF8  
\usepackage[russian]{babel}

\title{BridgeSystem}
\author{Максим Захаренков}
\date{July 2025}

\begin{document}

% \maketitle
\Large
\section{Открытия}
\begin{enumerate}
    \item[\cl{1}] - 11-22 от синглета треф или 16+ произвольный форсинг гейм.
    \item[\di{1}, \he{1}, \sp{1}] - 11-22 от пяти карт в масти открытия. Приоритет по старшинству.
    \item[1БК] - 15-17 равномерного расклада.
    \item[\cl{2}] - 5-10 от 4 треф в неравномере, обычно обещает не более 3 карт в черве и не более 6 карт в пике. В плохой зональности обычно требует 7 или меньше потерь, в остальных случаях не больше 8.
    \item[\di{2}] - 5-10 от 4 бубен в неравномере, обычно обещает не более 3 карт в трефе и не более 6 карт в пике. В плохой зональности обычно требует 7 или меньше потерь, в остальных случаях не больше 8.
    \item[\he{2}] - 5-10 от 4 черв в неравномере, обычно обещает не более 3 карт в бубне и не более 6 карт в пике. В плохой зональности обычно требует 7 или меньше потерь, в остальных случаях не больше 8.
    \item[\sp{2}] - 5-10 от 6 карт в пике. В плохой зональности обычно требует 7 или меньше потерь, в остальных случаях не больше 8.
    \item[2БК] - 20-22 равномерного расклада или \cl{1}\di{4}\he{4}\sp{4}.
    \item[\cl{3}, \di{3}, \he{3}, \sp{3}] - 5-10, от 7 карт в масти открытия, обещает не менее 3 фигур из 5.
    \item[$^*$] Все мастевые открытия на более высоких уровнях слабые, натуральные.
\end{enumerate}
\section{Продолжения после \cl{1}}
\begin{enumerate}
    \item[\di{1}] - 0-7 любые или 8+ от 5 карт в бубне.
    \item[\he{1}, \sp{1}] - 8+ от 4 карт натурально. При наличии от 4 карт в обоих мажорах при ровно четвёрке пик \he{1}, иначе --\sp{1}.
    \item[1БК] - 8-10, наименьший приоритет.
    \item[\cl{2}] - 10+, от 4 карт в трефе.
    \item[\di{2}, \he{2}, \sp{2}] - 12+, от 6 карт в масти, форсирует до уровня 3бк.
    \item[2БК] - 11-12, инвит, наименьший приоритет.
    \item[\cl{3}] - 6-9, от 6 карт в трефе. 
    \item[3БК, 6БК] - контракт.
    \item[4БК] - инвит на середину, \cl{3}\di{4}\he{3}\sp{3}.
    \item[5БК] - инвит на минимум,
    \cl{3}\di{4}\he{3}\sp{3}.
\end{enumerate}
\subsection{\cl{1} --- \di{1}}
\begin{enumerate}
    \item[\he{1}, \sp{1}] - 11-17 от трёх карт.
    \item[1БК] - 18-19 примерно равномерного расклада. Далее секвенции аналогичные открытию 1БК.
    \item[\cl{2}] - 11-17 от 5 карт в трефе. 5 карт только при \cl{5}\di{4}\he{2}\sp{2}. При 6 картах в трефе и четвёрке в мажоре приоритет у заявки \cl{2}.
    \item[\di{2}] - Открытие - форсинг гейм.
    \item[\he{2}, \sp{2}] - 19-22 \cl{4+}4+M(Реверс). 
    \item[2БК] - 19-22 \cl{4+}\di{4+}. Далее \cl{3},\di{3} - сайноффы, остальное натурально, форсирует.
    \item[\cl{3}] - 18-22 от 6 карт в трефе.
    \item[\di{3}, \he{3}, \sp{3}] - от 6 карт натурально (открытие форсинг гейм).
    \item[3БК] - от 6 карт в трефе (открытие форсинг гейм).
\end{enumerate}
\subsection{\cl{1} --- \he{1}}
\begin{enumerate}
    \item[\sp{1}] - 4 пик, форсирует.
    \item[1БК] - 11-14, нет 4 пик, нет 4 червей, нет 6 треф.
    \item[\cl{2}] - 11-14, от 6 треф.
    \item[\di{2}] - 15+ \cl{4+}\di{4+} (Реверс) или открытие форсинг гейм.
    \item[\he{2}] - 11-14, 4 червей
    \item[\sp{2}] - 15+ \cl{4+}\sp{4+} (Реверс).
    \item[2БК] - Многозначная заявка. Или 18-19 равномера, или фит и шлемовый интерес или 11-14 от 7 карт в трефе и 3-4 в черве.
    \item[\cl{3}] - 15-17, от 6 карт в трефе.
    \item[\he{3}] - 15-17 инвит с фитом, \cl{5}\di{2}\he{4}\sp{2}.
    \item[3БК] - 18-22, от 6 карт в трефе.
    \item[\sp{3}, \cl{4}, \di{4}] - слабый Сплинтер.
\end{enumerate}
\subsection{\cl{1} --- \sp{1}}
\begin{enumerate}
    \item[1БК] - 11-14, нет 4 пик, нет 6 треф.
    \item[\cl{2}] - 11-14, от 6 треф.
    \item[\di{2}] - 15+ \cl{4+}\di{4+} (Реверс) или открытие форсинг гейм.
    \item[\he{2}] - 15+ \cl{4+}\he{4+} (Реверс).
    \item[\sp{2}] - 11-14, 4 пик
    \item[2БК] - Многозначная заявка. Или 18-19 равномера, или фит и шлемовый интерес или 11-14 от 7 карт в трефе и 3-4 в пике.
    \item[\cl{3}] - 15-17, от 6 карт в трефе.
    \item[\sp{3}] - 15-17 инвит с фитом, \cl{5}\di{2}\he{2}\sp{4}.
    \item[3БК] - 18-22, от 6 карт в трефе.
    \item[\cl{4}, \di{4}, \he{4}] - слабый Сплинтер.
\end{enumerate}
\subsection{\cl{1} --- 1БК}
\begin{enumerate}
    \item[\cl{2}] - 11-14, от 6 треф.
    \item[\di{2}] - 15+ \cl{4+}\di{4+} (Реверс) или открытие форсинг гейм.
    \item[\he{2}] - 15+ \cl{4+}\he{4+} (Реверс).
    \item[\sp{2}] - 15+ \cl{4+}\sp{4+} (Реверс).
    \item[2БК] - 18-19 равномера.
    \item[\cl{3}] - 15-17, от 6 карт в трефе.
    \item[\di{3}, \he{3}, \sp{3}] - от 6 карт натурально (открытие форсинг гейм).
    \item[3БК] - 18-22, от 6 карт в трефе.
    \item[\cl{4}] - от 6 карт в трефе (открытие форсинг гейм).
\end{enumerate}
\subsection{\cl{1} --- \cl{2}}
\begin{enumerate}
    \item[\di{2}] - 11-12 или открытие форсинг гейм. Далее \he{2} автоматический вопрос.
    \item[\he{2}, \sp{2}] - 13+, показывают держки.
    \item[2БК] - нет фита, сильный форсинг.
    \item[\cl{3}] - от 4 треф, сильный форсинг.
    \item[\di{3}, \he{3}, \sp{3}] - от 6 карт натурально (открытие форсинг гейм).
    \item[3БК] - 13-14 равномерного расклада без фита.
\end{enumerate}
\subsubsection{\cl{1} --- \cl{2} --- \di{2} --- \he{2}}
\begin{enumerate}
    \item[\sp{2}] - открытие форсинг гейм. Далее 2БК автоматический вопрос с ответами:
    \begin{enumerate}
        \item[\cl{3}] - мажоры
        \item[\di{3}] - бубна с червой
        \item[\he{3}] - бубна с пикой
        \item[3БК] - 23+ равномерного расклада.
    \end{enumerate}
    \item[2БК] - 11-12, не более 3 треф.
    \item[\cl{3}] - 11-12, от 4 карт в трефе.
    \item[$^*$] - Есть идея разделить ФГ.
\end{enumerate}
\subsection{\cl{1} --- \di{2}}
\begin{enumerate}
    \item[\he{2}, \sp{2}] - от 6 карт натурально (открытие форсинг гейм).
    \item[2БК] - интерес к игре без козыря.
    \item[\cl{3}] - нет поддержки в бубне, есть своя длинная трефа.
    \item[\di{3}] - 0-1 карт в бубне, поиск контракта, нет желания играть без козыря.
    \item[\he{3}, \sp{3}, \cl{4}] - контроли 1 или 2 уровня на согласованной бубне.
    \item[3БК] - от 6 треф (открытие форсинг гейм).
    \item[\di{4}] - Блэквуд на бубне на червовом исключении.
    \item[\he{4}] - Блэквуд на бубне на пиковом исключении.
    \item[\sp{4}] - Блэквуд на бубне на трефовом исключении.
    \item[4БК] - 18-19 равномерного расклада. По каким-то причинам хотим передать лидерство партнёру.
\end{enumerate}
\subsection{\cl{1} --- \he{2}}
\begin{enumerate}
    \item[\sp{2}, \di{3}] - от 6 карт натурально (открытие форсинг гейм).
    \item[2БК] - интерес к игре без козыря.
    \item[\cl{3}] - нет поддержки в черве, есть своя длинная трефа.
    \item[\he{3}] - 0-1 карт в черве, поиск контракта, нет желания играть без козыря.
    \item[\sp{3}, \cl{4}, \di{4}] - контроли 1 или 2 уровня на согласованной черве.
    \item[3БК] - от 6 треф (открытие форсинг гейм).
    \item[\he{4}] - Блэквуд на черве на пиковом исключении.
    \item[\sp{4}] - Блэквуд на черве на трефовом исключении.
    \item[\cl{5}] - Блэквуд на черве на бубновом исключении.
    \item[4БК] - 18-19 равномерного расклада. По каким-то причинам хотим передать лидерство партнёру.
\end{enumerate}
\subsection{\cl{1} --- \sp{2}}
\begin{enumerate}
    \item[2БК] - интерес к игре без козыря.
    \item[\cl{3}] - нет поддержки в пике, есть своя длинная трефа.
    \item[\di{3}, \he{3}] - от 6 карт натурально (открытие форсинг гейм).
    \item[\sp{3}] - 0-1 карт в пике, поиск контракта, нет желания играть без козыря.
    \item[3БК] - от 6 треф (открытие форсинг гейм).
    \item[\cl{4}, \di{4}, \he{4}] - контроли 1 или 2 уровня на согласованной черве.
    \item[\sp{4}] - Блэквуд на пике на трефовом исключении.
    \item[\cl{5}] - Блэквуд на пике на бубновом исключении.
    \item[\di{5}] - Блэквуд на пике на червовом исключении.
    \item[4БК] - 18-19 равномерного расклада. По каким-то причинам хотим передать лидерство партнёру.
\end{enumerate}
\section{Продолжения после \di{1}}
\begin{enumerate}
    \item[\he{1}, \sp{1}] - (6)+, натурально от 4 карт, форсирует.
    \item[1БК] - 6-9, наименьший приоритет.
    \item[\cl{2}] - 10+, от 4 карт в трефе. Далее \di{2} - минимум, остальные заявки форсируют до уровня 3БК.
    \item[\di{2}] - 6-9 с фитом,
    \item[\he{2}, \sp{2}], 12+, от 6 карт натурально, форсируют до уровня 3БК.
    \item[2БК] - 10+, от 4 карт в бубне.
    \item[\cl{3}, \he{3}, \sp{3}] - 10+, от 4 карт в бубне, Сплинтер.
    \item[\di{3}] - 2-5, от 4 карт в бубне в неравномере.
    \item[3БК] - 13-15, расклада 4333 с рассеянными очками. Предложение контракта.
\end{enumerate}
\end{document}